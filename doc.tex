\documentclass{article}
\usepackage[tmargin=1.0in, lmargin=0.8in, rmargin=0.8in]{geometry}
\usepackage{floatrow}
%\DeclareUnicodeCharacter{2212}{-}
\usepackage{graphicx}
\usepackage{amsmath,bm, amsfonts}
\usepackage{fancyvrb}
\usepackage{hyperref}
\title{\huge Stochastic filtering\\ \large using nonlinear Vector Auto Regression}
\date{\today}
%\author{\normalsize Rahul O R}
\begin{document}
\maketitle
\vspace*{-1.cm}
\VerbatimInput{inputs.py}
\VerbatimInput{config.py}
\section*{Ridge regression of $y = \text{logit}(p)$}
For a detailed discussion, please refer :\\
\emph{Learning stochastic filtering} \(|\)
\href{https://arxiv.org/abs/2206.13018}{arXiv:2206.13018 } \(|\)
{{Rahul O. Ramakrishnan}}, {Andrea Auconi}, {Benjamin M. Friedrich}\\


\begin{itemize}
\item[$\circ$] From the noisy signal $u(t)$, the optimal filter produces the probability $p(t)$. i.e., $p \in (0,1)$
\item[$\circ$] Let 
\begin{equation}
y = \text{logit} (p) = \ln\left(\frac{p}{1-p} \right),
\end{equation}
where, $y \in (-\infty, \infty)$
\item[$\circ$] The nVAR algorithm in fact fit $u(t)$ into $y(t)$. This a ridge regression:
\begin{equation}
y = W \cdot \mathbb{O},
\end{equation}
where, $W$ is the weights that has to be optimized and $\mathbb{O}$ is the `feature vector' constructed using liner and non-linear combinations of delay-terms taken from $u$. The optimization process seeks for a global minima of the below loss function,
\begin{equation}
\text{loss} = \text{mse}(y \text{(true)}, \widehat{y}\text{(pred)}) + \beta \Vert W \Vert^2,
\end{equation}
where, $\beta$ is the ridge parameter that penalizing the size of weights.

\item[$\circ$] The predicted output $\widehat{y}$ is transformed to corresponding probability through a sigmoid function:
\begin{equation}
\widehat{p} = \frac{1}{1+ e^{-\widehat{y}}}
\end{equation}
\end{itemize}

\VerbatimInput{data/result.txt}

\begin{figure}[H]
    \centering
	\includegraphics[scale=0.9]{fig/dataset.pdf}
	\caption{open \texttt{view\textunderscore plot.html} for better view}
\end{figure}

\begin{figure}[H]
    \centering
	\includegraphics[scale=0.7]{fig/weight.pdf}
	\caption{learned weights}
\end{figure}
\pagebreak
\end{document}


